\documentclass[12pt,fleqn]{article}
%\documentclass[12pt,a4paper]{article}
\usepackage{natbib}
\usepackage{lineno}
%\usepackage{lscape}
%\usepackage{rotating}
%\usepackage{rotcapt, rotate}
\usepackage{amsmath,epsfig,epsf,psfrag}
\usepackage{setspace}
\usepackage{ulem}
\usepackage{xcolor}
\usepackage[labelfont=bf,labelsep=period]{caption} %for making figure and table numbers bold
\usepackage[colorlinks,bookmarksopen,bookmarksnumbered,citecolor=red,urlcolor=red]{hyperref}
\hypersetup{pdfpagemode=UseNone}

%\usepackage{a4wide,amsmath,epsfig,epsf,psfrag}


\def\be{{\ensuremath\mathbf{e}}}
\def\bx{{\ensuremath\mathbf{x}}}
\def\bX{{\ensuremath\mathbf{X}}}
\def\bthet{{\ensuremath\boldsymbol{\theta}}}
\newcommand{\VS}{V\&S}
\newcommand{\tr}{\mbox{tr}}
%\renewcommand{\refname}{\hspace{2.3in} \normalfont \normalsize LITERATURE CITED}
%this tells it to put 'Literature Cited' instead of 'References'
\bibpunct{(}{)}{,}{a}{}{;}
\oddsidemargin 0.0in
\evensidemargin 0.0in
\textwidth 6.5in
\headheight 0.0in
\topmargin 0.0in
\textheight=9.0in
%\renewcommand{\tablename}{\textbf{Table}}
%\renewcommand{\figurename}{\textbf{Figure}}
\renewcommand{\em}{\it}
\renewcommand\thefigure{S2.\arabic{figure}}
\renewcommand\thetable{S2.\arabic{table}}
\renewcommand\theequation{S2.\arabic{equation}}

\begin{document}


\rm \begin{flushleft}

\raggedbottom
\vspace{.5in}

\begin{center}
S2 Text: Full details of simulation study examining predictive extrapolation
\bigskip
\end{center}
\vspace{.3in}

\doublespacing

Spatially explicit statistical models are increasingly used to estimate animal abundance and predict species distributions from count data.  We employed a simulation study to examine the ability of the generalized independent variable hull (gIVH) to diagnose potential areas across the landscape where predictions of animal abundance using these models may be problematic, and to determine whether statistical inferences are more robust when restricted to sampling units within the gIVH.  Each simulation replicate consisted of several steps, including
\begin{enumerate}
  \item Simulate three hypothetical, statistically dependent, spatially autocorrelated environmental covariates over a 30$\times$30 grid,
  \item Simulate animal abundance across the landscape as a function of environmental covariates,
  \item Simulate animal population surveys across the landscape, including the position of count quadrats and the resulting animal counts,
  \item Estimate animal abundance as a function of two of the environmental covariates according three different models: a generalized linear model (GLM), a generalized additive model (GAM), and a spatio-temporal regression model (STRM).  For the latter, only a spatial dimension was modeled.  All such models were specified hierarchically, with MCMC used for posterior simulation.
  \item Calculate the gIVH using realized posterior variance (i.e. after data were collected and analyzed).
\end{enumerate}
We now describe each of these tasks and results of the simulation study in further detail before describing results.  All analyses were performed in the R programming environment \citep{RTeam2012}; requisite code to recreate analyses is available in the R package \texttt{SpatPred} that accompanies this article.

\underline{1. Simulating environmental covariates} \\
In the real world, different habitat covariates are often correlated with each other (e.g., altitude and precipitation), and are often patchily distributed across the landscape (i.e., are spatially autocorrelated).  We thus desired a procedure for generating covariates that would allow some level of statistical dependence among covariates, together with spatial autocorrelation.  For each simulation, we generated three spatially autocorrelated environmental covariates using a procedure motivated by linear coregionalization models in multivariate spatial statistics \citep[e.g.][]{GoulardVoltz1992} to impart desired behavior.   To start, we used the R package \texttt{RandomFields} to simulate 10 realizations ${\bf y}_i$ ($i \in 1,2,\hdots,10$) of independent, mean-zero random fields over a $30 \times 30$ grid (where the lower case bold type denotes a vector).  Each random field had a stationary, isotropic, exponential covariance structure, where the covariance $C$ between two survey units (i.e. grid cells) was a function of the distance $r$ between grid cell centroids, $C(r)=\exp(r/v)$. Note that the distance between horizontally and vertically adjacent grid cell centroids was standardized to 1.0; the scale parameter $v$ of the exponential covariance function for each random field was drawn from a Uniform(5,100) distribution to induce heterogeneity in the spatial scale of each process.

Next, we determined the values of three spatially autocorrelated habitat covariates, ${\bf z}_j$ by writing them as linear functions of the ${\bf y}_i$:
\begin{linenomath}
\begin{equation*}
  {\bf z}_j = \sum_i \omega_{ij} {\bf y}_i.
\end{equation*}
\end{linenomath}
Evidently, the covariance between each induced habitat covariate is a function of the weights $\omega_{ij}$.  We used the following strategy to set $\omega_{ij}$:
\begin{enumerate}
  \item Set all $\omega_{ij}=0$
  \item For each desired covariate, $j$, randomly sample four values ${\bf u}_j$ from the set $\{ 1,2,\hdots,10\}$ without replacement.
  \item For each value $u \in {\bf u}_j$, set $\omega_{uj} \sim \mathcal{N}(0,1)$.
\end{enumerate}
This procedure led to ``patchy" habitat covariates with realistic levels of covariation (Fig. \ref{fig:covs}).  To prevent redundancy and collinearity, we rejected and resampled covariate values (using the same procedure) whenever maximum absolute correlation among covariates was greater than 0.75.


\underline{2. Simulating animal abundance} \\

Given values of the three simulated covariates, we generated a vector of expected log-abundance in each cell ($\boldsymbol{\mu}$) as
\begin{linenomath}
\begin{equation*}
  \boldsymbol{\mu} = \beta_0 + {\bf X}\boldsymbol{\beta} + \boldsymbol{\epsilon},
\end{equation*}
\end{linenomath}
where the regression coefficients $\boldsymbol{\beta}$ were each drawn from a $\mathcal{N}(0,\tau)$ distribution, and the intercept, $\beta_0$, was drawn from a $\mathcal{N}(2.5,0.25)$ distribution.  The design matrix ${\bf X}$ was constructed assuming linear and quadratic effects for each covariate, together with all one-way interactions for a total of 9 coefficients in addition to the intercept.  The precision, $\tau$, was set to 2.5 for linear fixed effects, and to 5.0 for quadratic effects and one-way interaction terms. Residual Gaussian errors ($\boldsymbol{\epsilon}$) were drawn from a $\mathcal{N}(0,10)$ distribution.
Abundance in each cell $i$ was generated as
\begin{linenomath}
\begin{equation*}
  N_i \sim \text{Poisson}(\exp(\mu_i)).
\end{equation*}
\end{linenomath}
Regression coefficients were redrawn whenever $\sum_i N_i > 100,000$ or if the 20 most populous grid cells included $>90\%$ of total abundance to prevent unreasonably high or constricted distributions of abundance.

\underline{3. Simulating sample locations and count data} \\

For each simulated landscape, we selected 45 grid cells ($5\%$) for sampling.  We employed two possible survey designs: i) spatially balanced sampling using a random tessellation design \citep{StevensOlsen2004}, and ii) a convenience sample where the inclusion probability of each grid cell $i$ in the sampling frame was set proportional to $\exp(-0.2*r_i)$, where $r_i$ is the Euclidean distance between the centroid of grid cell $i$ and the center of the survey grid.  The latter survey design was meant to approximate the case where there is a ``base of operations" in the middle of the survey grid and more effort is expended close to the center due to simpler sampling logistics.  We configured simulations such that sample quadrats covered 10\% of each targeted grid cell, and generated animal counts for each quadrat $j$ as
\begin{linenomath}
\begin{equation*}
  C_j \sim \text{Binomial}(N_j,0.1).
\end{equation*}
\end{linenomath}

\underline{4. Estimating animal abundance} \\

For each set of count data, we attempted to estimate animal abundance over the landscape using three different hierarchical models, corresponding to a generalized linear model (GLM; linear, fixed effects of covariates on the log scale), a generalized additive model (GAM; smooth effects of covariates on the log scale), and a spatio-temporal regression model (with both linear, fixed effects and spatially autocorrelated random effects).  Further details on these models, together with the procedure used for posterior simulation, are presented in Text S1.  For this study, we used the RSR implementation of the STRM outlined in Text S1.  Prior distributions for each precision parameter, $\tau$, were set to $\text{Gamma}(1.0,0.01)$, which is diffuse while maintaining a flat shape near the origin.  Regression parameters ($\beta$) were given vague $\mathcal{N}(0,\tau=0.01)$ priors.


\underline{5. Calculating the gIVH} \\

As suggested in Text S1, we used Eq. S1.7 to calculate the gIVH, substituting in samples from the joint posterior distribution of each model for $\boldsymbol{\theta}$.


\renewcommand{\refname}{Literature Cited}
\bibliographystyle{ecology}

\bibliography{master_bib}

\end{flushleft}


\begin{figure}[!h]
\begin{center}
\includegraphics[width=6in]{Covs_coregion.pdf}
\end{center}
\caption{For each simulation, three spatially autocorrelated environmental covariates were generated via a linear coregionalization model.  Panels A-C show a single realization from this procedure, while panel D shows the distribution of sample correlations between two randomly selected covariates over 1000 simulations.}
\label{fig:covs}
\end{figure}




\end{document}





